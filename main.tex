
\let\negmedspace\undefined
\let\negthickspace\undefined
\documentclass[journal,12pt,twocolumn]{IEEEtran}
\usepackage{cite}
\usepackage{amsmath,amssymb,amsfonts,amsthm}
\usepackage{algorithmic}
\usepackage{graphicx}
\usepackage{textcomp}
\usepackage{xcolor}
\usepackage{txfonts}
\usepackage{listings}
\usepackage{enumitem}
\usepackage{mathtools}
\usepackage{gensymb}
\usepackage[breaklinks=true]{hyperref}
\usepackage{tkz-euclide} % loads  TikZ and tkz-base
\usepackage{listings}
%
%\usepackage{setspace}
%\usepackage{gensymb}
%\doublespacing
%\singlespacing

%\usepackage{graphicx}
%\usepackage{amssymb}
%\usepackage{relsize}
%\usepackage[cmex10]{amsmath}
%\usepackage{amsthm}
%\interdisplaylinepenalty=2500
%\savesymbol{iint}
%\usepackage{txfonts}
%\restoresymbol{TXF}{iint}
%\usepackage{wasysym}
%\usepackage{amsthm}
%\usepackage{iithtlc}
%\usepackage{mathrsfs}
%\usepackage{txfonts}
%\usepackage{stfloats}
%\usepackage{bm}
%\usepackage{cite}
%\usepackage{cases}
%\usepackage{subfig}
%\usepackage{xtab}
%\usepackage{longtable}
%\usepackage{multirow}
%\usepackage{algorithm}
%\usepackage{algpseudocode}
%\usepackage{enumitem}
%\usepackage{mathtools}
%\usepackage{tikz}
%\usepackage{circuitikz}
%\usepackage{verbatim}
%\usepackage{tfrupee}
%\usepackage{stmaryrd}
%\usetkzobj{all}
%    \usepackage{color}                                            %%
%    \usepackage{array}                                            %%
%    \usepackage{longtable}                                        %%
%    \usepackage{calc}                                             %%
%    \usepackage{multirow}                                         %%
%    \usepackage{hhline}                                           %%
%    \usepackage{ifthen}                                           %%
  %optionally (for landscape tables embedded in another document): %%
%    \usepackage{lscape}     
%\usepackage{multicol}
%\usepackage{chngcntr}
%\usepackage{enumerate}

%\usepackage{wasysym}
%\documentclass[conference]{IEEEtran}
%\IEEEoverridecommandlockouts
% The preceding line is only needed to identify funding in the first footnote. If that is unneeded, please comment it out.

\newtheorem{theorem}{Theorem}[section]
\newtheorem{problem}{Problem}
\newtheorem{proposition}{Proposition}[section]
\newtheorem{lemma}{Lemma}[section]
\newtheorem{corollary}[theorem]{Corollary}
\newtheorem{example}{Example}[section]
\newtheorem{definition}[problem]{Definition}
%\newtheorem{thm}{Theorem}[section] 
%\newtheorem{defn}[thm]{Definition}
%\newtheorem{algorithm}{Algorithm}[section]
%\newtheorem{cor}{Corollary}
\newcommand{\BEQA}{\begin{eqnarray}}
\newcommand{\EEQA}{\end{eqnarray}}
\newcommand{\define}{\stackrel{\triangle}{=}}
\theoremstyle{remark}
\newtheorem{rem}{Remark}

%\bibliographystyle{ieeetr}
\begin{document}
%

\bibliographystyle{IEEEtran}


\vspace{3cm}

\title{
Question 10.13.3.26
}
\author{Umair Parwez - EE22BTECH11054}	

\maketitle

\newpage

\bigskip

\renewcommand{\thefigure}{\theenumi}
\renewcommand{\thetable}{\theenumi}



\textbf{Question:} Two dice are thrown at the same time. Determine the probabiity that the difference
of the numbers on the two dice is 2. 
\\
\\
\textbf{Solution:}
Let X and Y represent the two dice.
Now, the PMF of X and Y are,
\begin{align}
	P_X(m) &= \frac{1}{6}, (0<m<7) \\
  &= 0, \text{otherwise} \\
  P_Y(m) &= \frac{1}{6}, (0<m<7) \\
  &= 0, \text{otherwise}
\end{align}

Let Z = X-Y. The PMF of Z is,
\begin{align}
  P_Z(k) &= P(Z=k)\\
  &= P(X-Y=k)\\
  &= P(X=k+Y)\\
  &= P_X(k+Y)\\
  &= \sum_{m=1}^{6} P_X(k+m)P_Y(m)\\
  &= \frac{1}{6} \sum_{m=1}^{6} P_X(k+m) \label{PMFZ}
\end{align}
Now,let E be the event that the difference of the two dice is 2. Then,
\begin{align}
  P(E) = P_Z(-2) + P_Z(2) 
\end{align}
Substituting $\eqref{PMFZ}$, 
\begin{align}
  P(E) &= \frac{1}{6} \sum_{m=1}^{6} P_X(m-2) + \frac{1}{6} \sum_{m=1}^{6} P_X(m+2)\\
  &= \frac{1}{6} \cdot \frac{2}{3} + \frac{1}{6} \cdot \frac{2}{3}\\
  &= \frac{2}{9}
\end{align}


\end{document}
